% !TeX encoding = windows-1251
% !TEX root = diplExample.tex

\chapter{Исходный код программы на Python}
\label{app:Python}

% Настраиваем оформление листинга для Python
\lstdefinestyle{Python}{
	language=Python,
	morekeywords={models, lambda, forms, self} % расширяем список ключевых слов
}

% Определяем ограничители для ввода меток на строки кода
% Для этого используем редкие для языка комбинации символов
\lstset{escapeinside={|@}{@|}}

Пример кода на Python 3 (взят с \href{http://python3.codes/alan-turings-automatic-machine/}{официального сайта}), реализующий симулятор машины Тьюринга для сложения унарных чисел (типа $11 + 111$).
Листинг позволяет делать (автоматическую) ссылку на какую"=нибудь строку. Например, на строку~\ref{lst:1} с командой \texttt{print(tape)}.

% Листинг кода на Python
%# Turing Machine simulator to add unary numbers (e.g. 11 + 111)
%#
\begin{lstlisting}[style=Python]
# prog is indexed by the current tape symbol (0 or 1) 
# and then by state (a kind of instruction pointer) 
# to get an 'instruction' comprising:
#   symbol to write on current tape position,
#   head action (-1 = move left, +1 = move right)
#   next state (like a goto jump).

#       symbol 0    symbol 1
prog = [[(1, +1, 1), (1, +1, 0)],         # state 0
		[(0, -1, 2), (1, +1, 1)],         # state 1
		[(0, +1, 2), (0, +1, 9)]]         # state 2
tape = [1,1,0,1,1,1,0,0,0]                # The data tape
head = 0                                  # head position on tape
state = 0                                 # instruction pointer
print(tape)  |@\label{lst:1}@|
while state != 9:                         # while not halt:
	symbol = tape[head]                   # read current tape symbol
	symbol, dir, state = t = prog[state][symbol] # lookup instruction
	print(' ' * (head * 3 + 1)+ '^  ' + str(t)) # display progress
	tape[head] = symbol                   # write new symbol on tape
	print(tape)
	head = head + dir                           # move tape head
\end{lstlisting}

