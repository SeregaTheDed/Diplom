 
\documentclass[a4paper,12pt]{diplom}
% \usepackage[latin1]{inputenc}
% \usepackage[utf8]{inputenc}
\inputencoding{utf8} % Кодировка вашего файла


\usepackage{paratype} % Шрифты (можно отключить, если дает ошибку)
%% Немного увеличим шрифт в математическом режиме, чтобы соответствовать размерам Paratype-шрифтов
\DeclareMathSizes{12}{13.4}{11}{10}

\usepackage[left=3cm,right=2cm,top=2cm,bottom=2cm]{geometry} % Размеры полей
\usepackage[onehalfspacing]{setspace} % Полуторный интервал
%\renewcommand{\baselinestretch}{1.25} % Полуторный интервал
\usepackage{indentfirst} % Абзацный отступ в начале разделов
\setlength{\parindent}{1.25cm} % Величина абзацного отступа

\usepackage[pdftex]{graphicx} % Для вставки изображений
\usepackage{array} % Для таблиц
\usepackage{booktabs} % Для красивых таблиц 
\usepackage{tikz} % Рисунки с помощью TikZ
\usepackage[linesnumbered,lined,ruled]{algorithm2e} % Для оформления псевдокода
%\usepackage{algorithm} % Альтернатива оформления псевдокода
%\usepackage{algpseudocode} % Альтернатива оформления псевдокода
\usepackage{listings} % Оформление листингов программ
\usepackage{icomma} % Удаляем тонкий пробел после запятой в мат. режиме

% Если на нумерованную формулу нет ссылки в тексте,
\mathtoolsset{showonlyrefs} % то она становится ненумерованной

% microtype улучшает распределение символов в строке
\usepackage{microtype}  % Можно отключить, если возникают ошибки компиляции

% Формируем PDF с полноценными перекрестными ссылками
\usepackage[unicode, pdfborder={0 0 0}, pdfstartview=FitV]{hyperref}

% Часто используемые макросы
\newcommand{\N}{\mathbb{N}}  % Множество натуральных чисел
\newcommand{\Z}{\mathbb{Z}}  % Множество целых чисел
\newcommand{\R}{\mathbb{R}}  % Множество действительных чисел
\DeclareMathOperator{\sgn}{sgn} % Знак числа
\DeclareMathOperator{\M}{\mathsf{M}} % Матожидание
\newcommand{\from}{\colon} % Двоеточие в определении функции. Пример: $f \from \R \to \N$.
% Заменяем англоязычные обозначения на русские
\renewcommand{\le}{\leqslant}
\renewcommand{\leq}{\leqslant}
\renewcommand{\ge}{\geqslant}
\renewcommand{\geq}{\geqslant}
\renewcommand{\emptyset}{\varnothing}
\renewcommand{\epsilon}{\varepsilon}


%%%%%%%%%%%%%%%%%%%%%%%%%%
% Конец преамбулы
%%%%%%%%%%%%%%%%%%%%%%%%%%

\begin{document}
	
	% Настраиваем общее для всех языков оформление листинга
	\lstset{
		%	breaklines=true,
		%	frame=l,
		%	showstringspaces=false,
		tabsize=4, % длина табуляции в пробелах
		formfeed=\newpage, % реакция на символ "form feed"
		extendedchars=true, % используем неанглийские буквы
		basicstyle=\ttfamily, % базовый стиль
		%	keywordstyle=\bfseries, % стиль ключевых слов (попробуйте \pmb если \bfseries не работает)
		commentstyle=\rmfamily\itshape, % стиль для комментариев
		stringstyle=\slshape, % стиль строк в кавычках
		numbers=left, % где проставляем номера строк; возможные значения: none, left, right
		numbersep=1em, % расстояние (по горизонтали) от номеров строк до кода
		stepnumber=1, % шаг отображения номеров строк. Если 1, то каждая строка помечается номером
		numberstyle=\footnotesize\color{black}, % стиль для номеров строк
	}
	
	
	% Содержимое титульного листа
	
	%\LetterHead{Минобр...}
	\Kafedra{Кафедра компьютерной безопасности и математических 
		методов обработки информации}
	
	% Зав. кафедрой
	\ZavKaf{Заведующий кафедрой,\\ к.\,ф.-м.\,н., доцент}{Мурин Д.М.}
	% Если это курсовая работа и виза зав. каф. не нужна, раскомментируйте следующую строку
	%\Kursovaya
	
	% Вид работы: Курсовая работа, Выпускная квалификационная работа, 
	\DocumentType{\large Выпускная квалификационная работа}
	
	% Название дипломной работы
	\Title{\begin{Large}\bfseries Анализ уязвимостей \\веб-приложений\end{Large}}
	
	% Направление подготовки
	\Napr{по направлению\\ 10.03.01 Информационная безопасность}
	
	% Руководитель
	\Chief{Научный руководитель\\ к.\,ф.-м.\,н., доцент}{~Власова О.\,В.}
	
	% Автор
	\Author{Студент группы ИБ-41БО}{С.\,И.~Штанько}
	
	%\City{Ярославль}
	%\Year{2017}
	
	
	% Создаем титульный лист
	\maketitle
	\chapter{Реферат}
	
	Объем \total{page} с., \total{chapternum} гл., \total{fignum} рис.,
	\total{tablenum} табл., \total{bibnum} источников, \total{appnum} прил.
	
	\medskip
	
	Ключевые слова: \textbf{Уязвимости веб-приложений, анализ безопасности сайтов}
	
	\medskip
	
	
	% Содержание
	\tableofcontents[Содержание]
	
	
	% Пример ненумерованной главы
	\chapternonum{Введение}
	
	
	Современный мир характеризуется стремительным развитием информационных технологий, все большей интеграцией цифровых решений в различные сферы жизни и деятельности. Веб-приложения стали неотъемлемой частью повседневности, обеспечивая доступ к услугам, информации и коммуникации. С ростом их популярности и сложности возрастает и актуальность обеспечения их безопасности. Уязвимости веб-приложений представляют собой лазейки, которые могут быть использованы злоумышленниками для нанесения ущерба пользователям, организациям и системам.
	
	Актуальность темы дипломной работы обусловлена возрастающей угрозой кибербезопасности, связанной с уязвимостями веб-приложений. Кибератаки становятся всё более изощрёнными и масштабными, а их последствия могут быть катастрофическими, приводя к утечке конфиденциальной информации, финансовым потерям и репутационному ущербу. 
	
	Цель дипломной работы – комплексное изучение и анализ наиболее распространенных уязвимостей веб-приложений, а также методов их обнаружения, предотвращения и устранения.
	
	Задачи дипломной работы:
	
	\begin{itemize}
		\item Рассмотреть основные виды уязвимостей веб-приложений, такие как инъекции SQL, межсайтовый скриптинг (XSS), подделка межсайтовых запросов (CSRF) и другие.
		\item Изучить причины возникновения уязвимостей, их потенциальные последствия и угрозы для безопасности веб-приложений.
		\item Проанализировать методы и инструменты для обнаружения и предотвращения уязвимостей веб-приложений.
		\item Изучить практические примеры и рекомендации по обеспечению безопасности веб-приложений.
		\item Провести анализ конкретных случаев уязвимостей и рассмотреть методы их устранения.
	\end{itemize}
	
	Объектом исследования дипломной работы являются веб-приложения, а предметом исследования – уязвимости веб-приложений и методы обеспечения их безопасности.
	
	Методологической основой дипломной работы служат методы анализа, синтеза, сравнения и обобщения информации из различных источников, включая научные статьи, техническую документацию, отчеты по безопасности и практические руководства.
	
	Практическая значимость дипломной работы заключается в возможности использования полученных знаний и рекомендаций для повышения безопасности веб-приложений, разработки защищенных программных продуктов и снижения рисков кибератак. 
	
	Результаты данной работы могут быть полезны разработчикам веб-приложений, специалистам по информационной безопасности, студентам и всем, кто интересуется вопросами кибербезопасности.
	
	
	% Если исходное название содержит специальные символы (например, \\),
	% то в квадратных скобках пишем упрощенный вариант названия для "Содержания".
	\chapter[Инъекции SQL]{Инъекции SQL}
	
	
	\section {Определение и примеры инъекций SQL.}
	
	
	\subsection{Сущность инъекции SQL}
	
	Инъекция SQL (SQL Injection) – это тип атаки, направленной на веб-приложения, использующие базы данных. Принцип ее действия заключается во внедрении вредоносного SQL-кода в поля ввода данных приложения. Цель такой атаки – исказить логику выполнения SQL-запросов, отправляемых к базе данных. В результате злоумышленник может получить несанкционированный доступ к чувствительным данным, манипулировать ими, нарушать работу приложения и даже получить полный контроль над сервером базы данных.
	
	
	\subsection{Механизм атаки}
	
	Для успешной реализации инъекции SQL злоумышленник должен воспользоваться уязвимостью в коде приложения, которая возникает при недостаточной проверке и обработке пользовательского ввода. 
	
	Этапы атаки:
	\begin{enumerate}[label=\arabic{enumi})]
		\item \textbf{Идентификация уязвимости: }
			Злоумышленник анализирует приложение, чтобы найти поля ввода, которые не проходят должную валидацию и фильтрацию данных.  Это могут быть формы авторизации, поиска, регистрации, комментарии и другие элементы взаимодействия с пользователем. 
		\item \textbf{Внедрение вредоносного кода: }
			Злоумышленник вводит в уязвимое поле специальным образом сформированный SQL-код. Этот код может быть предназначен для обхода аутентификации, извлечения данных, модификации записей или выполнения других вредоносных действий.
		\item \textbf{Исполнение измененного запроса: }
			Приложение, не распознавая вредоносный код, формирует SQL-запрос с учетом пользовательского ввода и отправляет его к базе данных. 
		\item \textbf{Получение доступа к данным или выполнение несанкционированных действий: }
			База данных выполняет измененный запрос, предоставляя злоумышленнику 	несанкционированный доступ к данным, возможность их модификации или выполнения других команд. 
	\end{enumerate}
	
	
	\subsection{Классификация инъекций SQL}
	
	По методу получения данных:
	\begin{enumerate}[label=\arabic{enumi})]
		\item \textbf{Внеполосные (Out-of-band): }
		Злоумышленник получает данные через внешний канал, отличный от используемого приложением.  Например, он может использовать функции базы данных для отправки данных на свой сервер или выполнения запросов к внешним ресурсам.
		\item \textbf{ Инференциальные (Inferential): }
		Злоумышленник анализирует ответы приложения на специально сформированные запросы, чтобы получить информацию о структуре базы данных, наличии определенных записей или содержимом данных.
		\item \textbf{Внедрение кода (Code Injection): }
		Злоумышленник вводит код, который будет выполнен на сервере приложения. Это может быть, например, код на языке базы данных (PL/SQL) или на языке программирования, используемом на сервере.
	\end{enumerate}
	
	По влиянию на приложение:
	\begin{enumerate}[label=\arabic{enumi})]
		\item \textbf{Логические: }
		Злоумышленник изменяет логику работы приложения, например, обходит аутентификацию, получает доступ к чужим данным или изменяет права доступа.
		\item \textbf{Извлечение данных: }
		Злоумышленник получает доступ к конфиденциальным данным, которые не должны быть ему доступны.
		\item \textbf{Модификация данных: }
		Злоумышленник изменяет или удаляет данные в базе данных.
	\end{enumerate}
	
	По способу внедрения вредоносного кода:
	\begin{enumerate}[label=\arabic{enumi})]
		\item \textbf{Ошибка синтаксиса: }
		Злоумышленник вводит символы, которые нарушают синтаксис SQL-запроса, что позволяет ему изменить логику его выполнения. 
		\item \textbf{Объединение запросов (Union-based): }
		Злоумышленник добавляет к исходному запросу еще один запрос, используя оператор UNION, чтобы получить доступ к данным из другой таблицы.
		\item \textbf{Внедрение подзапросов: }
		Злоумышленник внедряет подзапрос, который изменяет логику работы исходного запроса. 
	\end{enumerate}
	
	
	
	\subsection{Примеры инъекций SQL}
	
	\textbf{Пример 1 (Обход аутентификации):}\\
	Предположим, приложение использует следующий запрос для аутентификации пользователя:
	\begin{lstlisting}[ language=SQL, deletekeywords={IDENTITY},deletekeywords={[2]INT},morekeywords={clustered},framesep=8pt,xleftmargin=40pt,framexleftmargin=40pt,frame=tb,framerule=0pt]		
SELECT * FROM users 
WHERE username = '$username' AND password = '$password';
	\end{lstlisting}
	
	Злоумышленник может ввести в поле "username" значение `' OR '1'='1' --` .  В результате получится следующий запрос: 
	\begin{lstlisting}[ language=SQL, deletekeywords={IDENTITY},deletekeywords={[2]INT},morekeywords={clustered},framesep=8pt,xleftmargin=40pt,framexleftmargin=40pt,frame=tb,framerule=0pt]		
SELECT * FROM users 
WHERE username = '' OR '1'='1' --' AND password = '$password';
	\end{lstlisting}
	
	Поскольку условие `'1'='1'` всегда истинно, запрос вернет первую запись из таблицы `users`,  предоставляя злоумышленнику доступ к системе.
	
	\textbf{Пример 2 (Извлечение данных):}\\
	Предположим, приложение использует следующий запрос для поиска товара по его названию:
	\begin{lstlisting}[ language=SQL, deletekeywords={IDENTITY},deletekeywords={[2]INT},morekeywords={clustered},framesep=8pt,xleftmargin=40pt,framexleftmargin=40pt,frame=tb,framerule=0pt]		
SELECT * FROM products 
WHERE name LIKE '%$search_term%';
	\end{lstlisting}	
	Злоумышленник может ввести в поле поиска значение `'; SELECT * FROM users; --` .  В результате получится следующий запрос: 
	\begin{lstlisting}[ language=SQL, deletekeywords={IDENTITY},deletekeywords={[2]INT},morekeywords={clustered},framesep=8pt,xleftmargin=40pt,framexleftmargin=40pt,frame=tb,framerule=0pt]		
SELECT * FROM products 
WHERE name LIKE '%; SELECT * FROM users; --%'; 
	\end{lstlisting}	
	Этот запрос сначала выполнит поиск по таблице `products`, а затем извлечет все записи из таблицы `users`, предоставив злоумышленнику доступ к информации о пользователях.
	
	\textbf{Пример 3 (Модификация данных):}\\
	Предположим, приложение использует следующий запрос для обновления профиля пользователя:
	\begin{lstlisting}[ language=SQL, deletekeywords={IDENTITY},deletekeywords={[2]INT},morekeywords={clustered},framesep=8pt,xleftmargin=40pt,framexleftmargin=40pt,frame=tb,framerule=0pt]		
UPDATE users SET email = '$email' 
WHERE username = '$username';
	\end{lstlisting}
	
	Злоумышленник может ввести в поле "email" значение \\
	`'new-email'; UPDATE users SET isAdmin = 1 WHERE username = 'attacker'; --` . \\
	В результате получится следующий запрос:
	
	\begin{lstlisting}[ language=SQL, deletekeywords={IDENTITY},deletekeywords={[2]INT},morekeywords={clustered},framesep=8pt,xleftmargin=40pt,framexleftmargin=40pt,frame=tb,framerule=0pt]		
UPDATE users SET email = 'new-email'; 
UPDATE users SET isAdmin = 1
WHERE username = 'attacker'; --' WHERE username = '$username';
	\end{lstlisting}
	
	Этот запрос сначала обновит email пользователя, а затем установит флаг `isAdmin` для пользователя `attacker`,  предоставив ему права администратора. 
	Примеры инъекций SQL наглядно демонстрируют опасность данного типа атак и необходимость принятия мер для их предотвращения.
	
	
	
	
	
	
	
	
	\section{Причины возникновения инъекций SQL}
	
	Инъекции SQL остаются одной из наиболее распространенных и опасных уязвимостей веб-приложений.  Их возникновение обусловлено рядом факторов, связанных с ошибками проектирования, разработки и эксплуатации приложений.
	
	\subsection{Недостаточная валидация и фильтрация пользовательского ввода}
	
	Основная причина уязвимости к инъекциям SQL - недостаточная проверка и фильтрация данных, вводимых пользователем.  Приложения часто предполагают, что пользователи будут вводить только корректные и ожидаемые данные,  но злоумышленники могут использовать специальные символы и конструкции SQL-языка, чтобы изменить логику запросов к базе данных.  
	
	\textbf{Примеры недостаточной валидации:}
	
	\begin{enumerate}[label=\arabic{enumi})]
		\item Приложение не проверяет тип данных, например, ожидает число, но принимает строку, содержащую SQL-код.
		\item Приложение не ограничивает длину вводимых данных, что позволяет злоумышленнику ввести длинную строку с вредоносным кодом.
		\item Приложение не экранирует специальные символы, такие как кавычки, точки с запятой и другие, которые могут быть использованы для внедрения SQL-кода.
	\end{enumerate}
	
	\subsection{Использование динамических SQL-запросов}
	
	Динамические SQL-запросы формируются на основе пользовательского ввода, что делает их уязвимыми к инъекциям.  Если данные не проходят должную обработку перед включением в запрос, злоумышленник может внедрить свой код и изменить логику его выполнения. 
	
	\textbf{Пример динамического SQL-запроса:}
	
	\begin{lstlisting}[ language=SQL, deletekeywords={IDENTITY},deletekeywords={[2]INT},morekeywords={clustered},framesep=8pt,xleftmargin=40pt,framexleftmargin=40pt,frame=tb,framerule=0pt]    
String query = "SELECT * FROM users
WHERE username = '" +
username + 
"' AND password = '" + password + "'";
	\end{lstlisting}
	
	В этом примере, если пользователь введет в поле `username` значение `' OR '1'='1' --`,  то получится следующий запрос: 
	
	\begin{lstlisting}[ language=SQL, deletekeywords={IDENTITY},deletekeywords={[2]INT},morekeywords={clustered},framesep=8pt,xleftmargin=40pt,framexleftmargin=40pt,frame=tb,framerule=0pt]    
SELECT * FROM users 
WHERE username = '' OR '1'='1' --' AND password = '$password';
	\end{lstlisting}
	
	Этот запрос вернет все записи из таблицы `users`,  поскольку условие `'1'='1'`  всегда истинно. 
	
	\subsection{Неправильная конфигурация базы данных}
	
	Некорректные настройки базы данных также могут способствовать возникновению уязвимостей.  Например,  если учетная запись базы данных, используемая приложением, имеет избыточные привилегии,  то злоумышленник,  получив доступ к ней,  сможет выполнить больше вредоносных действий. 
	
	\textbf{Примеры неправильной конфигурации:}
	
	\begin{enumerate}[label=\arabic{enumi})]
		\item Учетная запись базы данных имеет права на выполнение команд, которые не требуются для работы приложения, например, на удаление таблиц.
		\item База данных разрешает выполнение пакетных запросов (batch queries),  что позволяет злоумышленнику выполнить несколько SQL-запросов в одном вызове. 
		\item База данных использует слабые пароли или хранит их в незашифрованном виде.
	\end{enumerate}
	
	\subsection{Использование устаревшего программного обеспечения}
	
	Устаревшее программное обеспечение,  включая базы данных и библиотеки для работы с ними, может содержать известные уязвимости, которые могут быть использованы для инъекций SQL.  Важно регулярно обновлять ПО, чтобы устранять уязвимости и минимизировать риски.
	
	\subsection{Недостаточное обучение и осведомленность разработчиков}
	
	Разработчики веб-приложений должны быть осведомлены о проблеме инъекций SQL и методах их предотвращения. Недостаточное обучение и понимание рисков могут привести к ошибкам в коде, которые сделают приложение уязвимым.
	
	\subsection{Использование небезопасных функций}
	
	Некоторые функции, предоставляемые базами данных и библиотеками, могут быть небезопасными, если их использовать неправильно. Например, функции для конкатенации строк могут быть использованы для внедрения SQL-кода, если пользовательский ввод не проходит предварительную обработку. 
	
	\textbf{Пример небезопасной функции:}
	
	\begin{lstlisting}[ language=SQL, deletekeywords={IDENTITY},deletekeywords={[2]INT},morekeywords={clustered},framesep=8pt,xleftmargin=40pt,framexleftmargin=40pt,frame=tb,framerule=0pt]    
String query = "SELECT * FROM users 
WHERE username = '" + escape(username) + "'";
	\end{lstlisting}
	
	В этом примере функция `escape()`  может быть небезопасной, если она не экранирует все специальные символы. 
	
	\subsection{Человеческий фактор}
	
	Ошибки разработчиков,  небрежность при настройке сервера базы данных или недостаточная осведомленность пользователей о рисках могут привести к уязвимостям и успешным атакам.
	
	\textbf{Примеры человеческого фактора:}
	
	\begin{enumerate}[label=\arabic{enumi})]
		\item Разработчик забыл проверить пользовательский ввод перед использованием его в SQL-запросе.
		\item Администратор базы данных установил слабый пароль для учетной записи приложения. 
		\item Пользователь открыл фишинговое письмо и ввел свои учетные данные на поддельном сайте. 
	\end{enumerate}
	
	 Учитывая многообразие причин возникновения инъекций SQL,  важно применять комплексный подход к обеспечению безопасности веб-приложений,  который включает в себя меры по предотвращению,  обнаружению и устранению уязвимостей.
	 
	 
	 
	 
	 
	 
	 
	 
	 \section{Возможные последствия и угрозы для приложений и баз данных}
	 
	 Успешная инъекция SQL может иметь серьезные последствия для веб-приложений и баз данных, приводя к нарушению конфиденциальности, целостности и доступности данных, а также к финансовым и репутационным потерям. 
	 
	 \subsection{Нарушение конфиденциальности данных}
	 
	 Злоумышленники могут использовать инъекции SQL для получения несанкционированного доступа к конфиденциальным данным, таким как:
	 
	 \begin{itemize}
	 	\item \textbf{Персональные данные пользователей:} имена, адреса, номера телефонов, паспорта, данные кредитных карт и другая чувствительная информация.
	 	\item \textbf{Финансовая информация:} данные о банковских счетах, транзакциях, платежах и другие финансовые сведения.
	 	\item \textbf{Коммерческая тайна:} информация о продуктах, услугах, партнерах, клиентах и другие конфиденциальные данные компании. 
	 \end{itemize}
	 
	 Утечка таких данных может привести к мошенничеству, шантажу, краже личных данных и другим серьезным последствиям.
	 
	 \subsection{Нарушение целостности данных}
	 
	 Инъекции SQL могут быть использованы для модификации или удаления данных в базе данных, что может привести к:
	 
	 \begin{itemize}
	 	\item \textbf{Искажению информации:} злоумышленник может изменить данные о пользователях, продуктах, заказах и т. д., что приведет к ошибкам в работе приложения и принятию неверных решений.
	 	\item \textbf{Потере данных:} злоумышленник может удалить важные данные, что приведет к нарушению работы приложения и финансовым потерям.
	 	\item \textbf{Внедрению ложной информации:} злоумышленник может добавить в базу данных ложные записи, что может быть использовано для мошенничества или дезинформации. 
	 \end{itemize}
	 
	 \subsection{Нарушение доступности данных}
	 
	 Инъекции SQL могут быть использованы для нарушения работы приложения и базы данных, что приведет к:
	 
	 \begin{itemize}
	 	\item \textbf{Отказу в обслуживании (DoS):} злоумышленник может отправить запрос, который приведет к высокой нагрузке на сервер базы данных, что сделает приложение недоступным для пользователей.
	 	\item \textbf{Повреждению базы данных:} злоумышленник может выполнить команды, которые приведут к повреждению структуры базы данных или ее файлов, что сделает данные недоступными.
	 	\item \textbf{Взлому сервера:} злоумышленник может использовать инъекцию SQL для получения доступа к серверу базы данных и дальнейшего его взлома.
	 \end{itemize} 
	 
	 \subsection{Финансовые потери}
	 
	 Последствия инъекций SQL могут привести к значительным финансовым потерям для компании, включая:
	 
	 \begin{itemize}
	 	\item \textbf{Ущерб от мошенничества:} злоумышленники могут использовать украденные данные для совершения мошеннических операций, что приведет к финансовым потерям для компании и ее клиентов. 
	 	\item \textbf{Штрафы за нарушение законодательства:} утечка персональных данных может привести к штрафам со стороны регулирующих органов. 
	 	\item \textbf{Расходы на восстановление данных и системы:} восстановление базы данных и системы после атаки может потребовать значительных финансовых затрат. 
	 	\item \textbf{Потеря доверия клиентов:} инциденты с безопасностью могут привести к потере доверия клиентов и снижению репутации компании. 
	 \end{itemize}
	 
	 \subsection{Репутационные потери}
	 
	 Инциденты с безопасностью, связанные с инъекциями SQL, могут серьезно повредить репутации компании.  Клиенты могут потерять доверие к компании, что приведет к снижению продаж и потере доли рынка.  Кроме того, инциденты с безопасностью могут негативно повлиять на отношения с партнерами и инвесторами. 
	 
	 \subsection{Примеры реальных инцидентов}
	 
	 \begin{itemize}
	 	\item \textbf{Взлом компании TalkTalk (2015):} Хакеры использовали инъекцию SQL для получения доступа к данным 157 000 клиентов, включая имена, адреса, даты рождения, номера телефонов и данные банковских карт. Компания была оштрафована на £400 000 за нарушение законодательства о защите данных.
	 	\item \textbf{Взлом сайта Freepik (2020): } Хакеры использовали инъекцию SQL для получения доступа к внутреннему серверу базы данных с пользовательскими данными. Были похищены данные более 8 млн. пользователей. Сайт понес огромные финансовые и репутационные потери. 
	 \end{itemize}
	 
	 \textbf{Примеры реальных инцидентов наглядно демонстрируют серьезность угрозы,  которую представляют инъекции SQL.  Поэтому важно применять эффективные меры для защиты веб-приложений и баз данных от данного типа атак.} 
	 
	 
	 
	 
	 
	 
	 
	 
	 
	 \section{Методы обнаружения и предотвращения инъекций SQL}
	 
	 Для обеспечения безопасности веб-приложений и защиты от инъекций SQL необходимо применять комплекс мер,  включающий методы обнаружения уязвимостей,  предотвращения атак и минимизации их последствий. 
	 
	 \subsection{Методы обнаружения уязвимостей}
	 
	 Существует несколько методов обнаружения уязвимостей к инъекциям SQL:
	 
	 \begin{itemize}
	 	\item \textbf{Ручное тестирование:} опытные специалисты по безопасности могут вручную анализировать код приложения и проводить тесты на наличие уязвимостей.  Этот метод требует высокой квалификации и может быть трудоемким. 
	 	\item \textbf{Автоматизированное сканирование:} существуют специальные инструменты для автоматического сканирования веб-приложений на наличие уязвимостей,  включая инъекции SQL.  Эти инструменты могут быть полезны для быстрого выявления потенциальных проблем,  но они не всегда могут обнаружить все уязвимости. 
	 	\item \textbf{Анализ исходного кода:} анализ исходного кода приложения может помочь выявить участки кода, которые могут быть уязвимы к инъекциям SQL.  Этот метод требует доступа к исходному коду и может быть сложным для больших приложений. 
	 	\item \textbf{Фаззинг:} фаззинг - это метод тестирования, при котором приложению отправляются случайные или специально сформированные данные для выявления ошибок и уязвимостей.  Фаззинг может быть эффективным методом обнаружения уязвимостей к инъекциям SQL. 
	 \end{itemize}
	 
	 \subsection{Методы предотвращения атак}
	 
	 Существует несколько методов предотвращения атак с использованием инъекций SQL:
	 
	 \begin{itemize}
	 	\item \textbf{Валидация и фильтрация пользовательского ввода:} все данные, полученные от пользователя, должны быть проверены на соответствие ожидаемому типу, формату и длине.  Специальные символы, которые могут быть использованы для внедрения SQL-кода,  должны быть экранированы или удалены. 
	 	\item \textbf{Использование параметризованных запросов:} параметризованные запросы - это способ создания SQL-запросов, при котором пользовательский ввод передается отдельно от самого запроса.  Это позволяет избежать внедрения SQL-кода, так как база данных обрабатывает пользовательский ввод как данные, а не как часть запроса. 
	 	\item \textbf{Использование хранимых процедур:} хранимые процедуры - это предварительно скомпилированные SQL-запросы, которые хранятся на сервере базы данных.  Использование хранимых процедур может помочь предотвратить инъекции SQL, так как они ограничивают возможности злоумышленника по изменению логики запроса. 
	 	\item \textbf{Принцип наименьших привилегий:} учетные записи базы данных, используемые приложением, должны иметь минимальные необходимые привилегии для выполнения своих функций.  Это ограничивает возможности злоумышленника, если он получит доступ к учетной записи. 
	 	\item \textbf{Регулярное обновление программного обеспечения:} устаревшее программное обеспечение может содержать известные уязвимости, которые могут быть использованы для инъекций SQL.  Важно регулярно обновлять ПО, чтобы устранять уязвимости и минимизировать риски. 
	 \end{itemize}
	 
	 \subsection{Минимизация последствий}
	 
	 Даже при применении мер по предотвращению атак, всегда есть вероятность, что злоумышленник сможет найти и использовать уязвимость.  Поэтому важно принимать меры для минимизации последствий успешных атак:
	 
	 \begin{itemize}
	 	\item \textbf{Мониторинг и аудит:} регулярный мониторинг активности базы данных и аудит журналов событий может помочь выявить подозрительную активность и своевременно реагировать на атаки. 
	 	\item \textbf{Резервное копирование данных:} регулярное резервное копирование данных позволяет восстановить информацию в случае ее потери или повреждения. 
	 	\item \textbf{Планы реагирования на инциденты:} компания должна иметь план реагирования на инциденты безопасности, который определяет действия, которые необходимо предпринять в случае атаки. 
	 \end{itemize}
	 
	 \textbf{Применение комплексного подхода к обнаружению, предотвращению и минимизации последствий инъекций SQL является ключевым фактором для обеспечения безопасности веб-приложений и баз данных.} 
	 
	 
	 
	 
	 
	 
	 
	 
	 \section{Инструменты и техники для защиты от инъекций SQL}
	 
	 Существует множество инструментов и техник, которые помогают разработчикам и специалистам по безопасности защитить веб-приложения от инъекций SQL.  Рассмотрим некоторые из наиболее распространенных и эффективных вариантов.
	 
	 \subsection{Инструменты статического анализа кода}
	 
	 Статический анализ кода - это процесс автоматического анализа исходного кода программы для выявления потенциальных ошибок и уязвимостей.  Существуют инструменты статического анализа кода,  специально предназначенные для обнаружения уязвимостей к инъекциям SQL:
	 
	 \begin{itemize}
	 	\item \textbf{FindBugs}:  бесплатный инструмент с открытым исходным кодом, который анализирует Java-код и выявляет различные уязвимости,  включая инъекции SQL.
	 	\item \textbf{Yasca}:  еще один бесплатный инструмент с открытым исходным кодом, который поддерживает анализ кода на нескольких языках программирования,  включая Java,  Python и C++.
	 	\item \textbf{SonarQube}:  платформа для статического анализа кода,  которая предлагает широкий спектр функций,  включая обнаружение инъекций SQL,  анализ качества кода и управление техническим долгом. 
	 \end{itemize}
	 
	 \subsection{Инструменты динамического анализа приложений}
	 
	 Динамический анализ приложений (DAST) - это процесс тестирования безопасности работающего приложения.  DAST-инструменты могут автоматически отправлять запросы к приложению и анализировать ответы для выявления уязвимостей,  включая инъекции SQL:
	 
	 \begin{itemize}
	 	\item \textbf{OWASP ZAP}:  бесплатный инструмент с открытым исходным кодом,  который предлагает широкий спектр функций для динамического анализа веб-приложений,  включая сканирование на наличие инъекций SQL,  межсайтовый скриптинг (XSS) и другие уязвимости. 
	 	\item \textbf{Burp Suite}:  коммерческий инструмент, который предлагает продвинутые функции для ручного и автоматизированного тестирования безопасности веб-приложений.  Он включает в себя сканер уязвимостей,  перехватчик запросов и другие инструменты,  полезные для обнаружения и эксплуатации инъекций SQL.
	 	\item \textbf{Acunetix}:  коммерческий сканер уязвимостей, который специализируется на автоматическом обнаружении широкого спектра уязвимостей веб-приложений,  включая инъекции SQL,  XSS и CSRF. 
	 \end{itemize}
	 
	 \subsection{Фаззеры}
	 
	 Фаззеры - это инструменты, которые автоматически генерируют и отправляют некорректные или неожиданные данные в приложение для выявления ошибок и уязвимостей.  Фаззеры могут быть эффективными для обнаружения уязвимостей к инъекциям SQL:
	 
	 \begin{itemize}
	 	\item \textbf{Sulley Fuzzing Framework}:  бесплатный инструмент с открытым исходным кодом. Он был специально разработан для работы в течении нескольких дней и автоматизации процесса фаззинга. 
	 	\item \textbf{wfuzz}:  универсальный фаззер,  который может быть использован для тестирования различных протоколов и приложений,  включая веб-приложения.  Он может быть настроен для обнаружения инъекций SQL путем отправки специально сформированных запросов. 
	 \end{itemize}
	 
	 \subsection{Техники безопасного программирования}
	 
	 Помимо использования инструментов,  разработчики должны применять техники безопасного программирования,  чтобы предотвратить инъекции SQL:
	 
	 \begin{itemize}
	 	\item \textbf{Валидация и фильтрация пользовательского ввода:}  всегда проверяйте данные,  полученные от пользователя,  на соответствие ожидаемому типу,  формату и длине.  Специальные символы должны быть экранированы или удалены. 
	 	\item \textbf{Использование параметризованных запросов:}  всегда используйте параметризованные запросы вместо конкатенации строк для создания SQL-запросов. 
	 	\item \textbf{Использование хранимых процедур:}  рассмотрите возможность использования хранимых процедур для выполнения типовых операций с базой данных. 
	 	\item \textbf{Принцип наименьших привилегий:}  предоставляйте учетным записям базы данных минимальные необходимые привилегии. 
	 	\item \textbf{Регулярное обновление ПО:}  обновляйте программное обеспечение,  используемое в приложении,  включая библиотеки и фреймворки,  до последних версий,  чтобы устранять известные уязвимости. 
	 \end{itemize}
	 	
	 \textbf{Применение этих инструментов и техник поможет разработчикам создавать более безопасные веб-приложения и защитить их от инъекций SQL.} 
	 	
	 
	 
	 
	 
	 
	 
	
	
	
	
	
	
	
	
	
	
	
	
	
	
	
	% В заключении подводятся итоги выполненной работы, рассказывается о~том, что удалось и~что не~удалось сделать, описываются перспективы продолжения исследований.
	\renewcommand\bibname{Список литературы}
	\bibliographystyle{ugost2003}
	\bibliography{bibliography}
	
	
	
	% Конец документа
\end{document}