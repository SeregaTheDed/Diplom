 
\documentclass[a4paper,12pt]{diplom}
% \usepackage[latin1]{inputenc}
% \usepackage[utf8]{inputenc}
\inputencoding{utf8} % Кодировка вашего файла


\usepackage{paratype} % Шрифты (можно отключить, если дает ошибку)
%% Немного увеличим шрифт в математическом режиме, чтобы соответствовать размерам Paratype-шрифтов
\DeclareMathSizes{12}{13.4}{11}{10}

\usepackage[left=3cm,right=2cm,top=2cm,bottom=2cm]{geometry} % Размеры полей
\usepackage[onehalfspacing]{setspace} % Полуторный интервал
%\renewcommand{\baselinestretch}{1.25} % Полуторный интервал
\usepackage{indentfirst} % Абзацный отступ в начале разделов
\setlength{\parindent}{1.25cm} % Величина абзацного отступа

\usepackage[pdftex]{graphicx} % Для вставки изображений
\usepackage{array} % Для таблиц
\usepackage{booktabs} % Для красивых таблиц 
\usepackage{tikz} % Рисунки с помощью TikZ
\usepackage[linesnumbered,lined,ruled]{algorithm2e} % Для оформления псевдокода
%\usepackage{algorithm} % Альтернатива оформления псевдокода
%\usepackage{algpseudocode} % Альтернатива оформления псевдокода
\usepackage{listings} % Оформление листингов программ
\usepackage{icomma} % Удаляем тонкий пробел после запятой в мат. режиме

% Если на нумерованную формулу нет ссылки в тексте,
\mathtoolsset{showonlyrefs} % то она становится ненумерованной

% microtype улучшает распределение символов в строке
\usepackage{microtype}  % Можно отключить, если возникают ошибки компиляции

% Формируем PDF с полноценными перекрестными ссылками
\usepackage[unicode, pdfborder={0 0 0}, pdfstartview=FitV]{hyperref}

% Часто используемые макросы
\newcommand{\N}{\mathbb{N}}  % Множество натуральных чисел
\newcommand{\Z}{\mathbb{Z}}  % Множество целых чисел
\newcommand{\R}{\mathbb{R}}  % Множество действительных чисел
\DeclareMathOperator{\sgn}{sgn} % Знак числа
\DeclareMathOperator{\M}{\mathsf{M}} % Матожидание
\newcommand{\from}{\colon} % Двоеточие в определении функции. Пример: $f \from \R \to \N$.
% Заменяем англоязычные обозначения на русские
\renewcommand{\le}{\leqslant}
\renewcommand{\leq}{\leqslant}
\renewcommand{\ge}{\geqslant}
\renewcommand{\geq}{\geqslant}
\renewcommand{\emptyset}{\varnothing}
\renewcommand{\epsilon}{\varepsilon}


%%%%%%%%%%%%%%%%%%%%%%%%%%
% Конец преамбулы
%%%%%%%%%%%%%%%%%%%%%%%%%%

\begin{document}
	
	% Содержимое титульного листа
	
	%\LetterHead{Минобр...}
	\Kafedra{Кафедра компьютерной безопасности и математических 
		методов обработки информации}
	
	% Зав. кафедрой
	\ZavKaf{Заведующий кафедрой,\\ к.\,ф.-м.\,н., доцент}{Мурин Д.М.}
	% Если это курсовая работа и виза зав. каф. не нужна, раскомментируйте следующую строку
	%\Kursovaya
	
	% Вид работы: Курсовая работа, Выпускная квалификационная работа, 
	\DocumentType{\large Выпускная квалификационная работа}
	
	% Название дипломной работы
	\Title{\begin{Large}\bfseries Анализ уязвимостей \\веб-приложений\end{Large}}
	
	% Направление подготовки
	\Napr{по направлению\\ 10.03.01 Информационная безопасность}
	
	% Руководитель
	\Chief{Научный руководитель\\ к.\,ф.-м.\,н., доцент}{~Власова О.\,В.}
	
	% Автор
	\Author{Студент группы ИБ-41БО}{С.\,И.~Штанько}
	
	%\City{Ярославль}
	%\Year{2017}
	
	
	% Создаем титульный лист
	\maketitle
	\chapter{Реферат}
	
	Объем \total{page} с., \total{chapternum} гл., \total{fignum} рис.,
	\total{tablenum} табл., \total{bibnum} источников, \total{appnum} прил.
	
	\medskip
	
	Ключевые слова: \textbf{Уязвимости веб-приложений, анализ безопасности сайтов}
	
	\medskip
	
	
	% Содержание
	\tableofcontents[Содержание]
	
	
	% Пример ненумерованной главы
	\chapternonum{Введение}
	
	
	Современный мир характеризуется стремительным развитием информационных технологий, все большей интеграцией цифровых решений в различные сферы жизни и деятельности. Веб-приложения стали неотъемлемой частью повседневности, обеспечивая доступ к услугам, информации и коммуникации. С ростом их популярности и сложности возрастает и актуальность обеспечения их безопасности. Уязвимости веб-приложений представляют собой лазейки, которые могут быть использованы злоумышленниками для нанесения ущерба пользователям, организациям и системам.
	
	Актуальность темы дипломной работы обусловлена возрастающей угрозой кибербезопасности, связанной с уязвимостями веб-приложений. Кибератаки становятся всё более изощрёнными и масштабными, а их последствия могут быть катастрофическими, приводя к утечке конфиденциальной информации, финансовым потерям и репутационному ущербу. 
	
	Цель дипломной работы – комплексное изучение и анализ наиболее распространенных уязвимостей веб-приложений, а также методов их обнаружения, предотвращения и устранения.
	
	Задачи дипломной работы:
	
	\begin{itemize}
		\item Рассмотреть основные виды уязвимостей веб-приложений, такие как инъекции SQL, межсайтовый скриптинг (XSS), подделка межсайтовых запросов (CSRF) и другие.
		\item Изучить причины возникновения уязвимостей, их потенциальные последствия и угрозы для безопасности веб-приложений.
		\item Проанализировать методы и инструменты для обнаружения и предотвращения уязвимостей веб-приложений.
		\item Изучить практические примеры и рекомендации по обеспечению безопасности веб-приложений.
		\item Провести анализ конкретных случаев уязвимостей и рассмотреть методы их устранения.
	\end{itemize}
	
	Объектом исследования дипломной работы являются веб-приложения, а предметом исследования – уязвимости веб-приложений и методы обеспечения их безопасности.
	
	Методологической основой дипломной работы служат методы анализа, синтеза, сравнения и обобщения информации из различных источников, включая научные статьи, техническую документацию, отчеты по безопасности и практические руководства.
	
	Практическая значимость дипломной работы заключается в возможности использования полученных знаний и рекомендаций для повышения безопасности веб-приложений, разработки защищенных программных продуктов и снижения рисков кибератак. 
	
	Результаты данной работы могут быть полезны разработчикам веб-приложений, специалистам по информационной безопасности, студентам и всем, кто интересуется вопросами кибербезопасности.
	
	
	% Если исходное название содержит специальные символы (например, \\),
	% то в квадратных скобках пишем упрощенный вариант названия для "Содержания".
	\chapter[Инъекции SQL]{Инъекции SQL}
	
	
	\section {Определение и примеры инъекций SQL.}
	
	
	\subsection{Сущность инъекции SQL}
	
	Инъекция SQL (SQL Injection) – это тип атаки, направленной на веб-приложения, использующие базы данных. Принцип ее действия заключается во внедрении вредоносного SQL-кода в поля ввода данных приложения. Цель такой атаки – исказить логику выполнения SQL-запросов, отправляемых к базе данных. В результате злоумышленник может получить несанкционированный доступ к чувствительным данным, манипулировать ими, нарушать работу приложения и даже получить полный контроль над сервером базы данных.
	
	
	\subsection{Механизм атаки}
	
	
	
	
	\subsection{Классификация инъекций SQL }
	
	
	
	
	\subsection{Механизм атаки}
	
	
	
	
	\chapternonum{Заключение}
	
	% В заключении подводятся итоги выполненной работы, рассказывается о~том, что удалось и~что не~удалось сделать, описываются перспективы продолжения исследований.
	\renewcommand\bibname{Список литературы}
	\bibliographystyle{ugost2003}
	\bibliography{bibliography}
	
	
	
	% Конец документа
\end{document}